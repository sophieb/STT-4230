\documentclass[ignorenonframetext,]{beamer}

\usepackage{array} % Pour faire des tableaux dont les colonnes ont une largeur fixe
\newcolumntype{L}[1]{>{\raggedright\let\newline\\\arraybackslash\hspace{0pt}}m{#1}}
\newcolumntype{C}[1]{>{\centering\let\newline\\\arraybackslash\hspace{0pt}}m{#1}}
\newcolumntype{R}[1]{>{\raggedleft\let\newline\\\arraybackslash\hspace{0pt}}m{#1}}


\usepackage{amssymb,amsmath}
\usepackage{ifxetex,ifluatex}
\usepackage{fixltx2e} % provides \textsubscript
\usepackage{lmodern}
\ifxetex
  \usepackage{fontspec,xltxtra,xunicode}
  \defaultfontfeatures{Mapping=tex-text,Scale=MatchLowercase}
  \newcommand{\euro}{€}
\else
  \ifluatex
    \usepackage{fontspec}
    \defaultfontfeatures{Mapping=tex-text,Scale=MatchLowercase}
    \newcommand{\euro}{€}
  \else
    \usepackage[T1]{fontenc}
    \usepackage[utf8]{inputenc}
      \fi
\fi
\IfFileExists{upquote.sty}{\usepackage{upquote}}{}
% use microtype if available
\IfFileExists{microtype.sty}{\usepackage{microtype}}{}
\usepackage{longtable,booktabs}
\usepackage{caption}
% These lines are needed to make table captions work with longtable:
\makeatletter
\def\fnum@table{\tablename~\thetable}
\makeatother

% Comment these out if you don't want a slide with just the
% part/section/subsection/subsubsection title:
\AtBeginPart{
  \let\insertpartnumber\relax
  \let\partname\relax
  \frame{\partpage}
}
\AtBeginSection{
  \let\insertsectionnumber\relax
  \let\sectionname\relax
  \frame{\sectionpage}
}
\AtBeginSubsection{
  \let\insertsubsectionnumber\relax
  \let\subsectionname\relax
  \frame{\subsectionpage}
}

\setlength{\parindent}{0pt}
\setlength{\parskip}{6pt plus 2pt minus 1pt}
\setlength{\emergencystretch}{3em}  % prevent overfull lines
\providecommand{\tightlist}{%
	\setlength{\itemsep}{0pt}\setlength{\parskip}{0pt}}
\setcounter{secnumdepth}{0}

\title{Synthèse calculs de base en R}
\author{Sophie Baillargeon, Université Laval}

\begin{document}

\begin{frame}{Fonctions et opérateurs de base pour le calcul mathématique ou de statistiques descriptives}

\begin{columns}
\column{\dimexpr\paperwidth-25pt}
\begin{scriptsize}
\begin{tabular}{L{6em}|L{11em}L{5em}L{6em}L{4.5em}} \hline
Calcul & opère de façon vectorielle & combine, retourne une valeur & combine, retourne valeur(s) & combine, retourne un vecteur \\ \hline \hline

opérations arithmétiques & \texttt{+}, \texttt{-}, \texttt{*}, \texttt{/}, \texttt{\^{}}, \texttt{\%\%}, \texttt{\%/\%} & \texttt{sum}, \texttt{prod} & & \texttt{cumsum},  \texttt{cumprod}, \texttt{diff} \\ \hline

logarithme et exponentiel, trigonométrie, arrondissement, etc. & \texttt{log}, \texttt{log10}, \texttt{log2}, \texttt{exp}, \texttt{sin}, \texttt{cos}, \texttt{tan}, \texttt{acos}, \texttt{asin}, \texttt{atan}, \texttt{atan2}, \texttt{ceiling}, \texttt{floor},  \texttt{round}, \texttt{trunc}, \texttt{signif}, \texttt{sqrt}, \texttt{abs}, \texttt{sign}, \texttt{beta}, \texttt{gamma}, \texttt{choose}, \texttt{factorial}, etc. & & & \\ \hline \hline
	
mesures de & \texttt{pmax}, \texttt{pmin} & \texttt{max}, \texttt{min} & \texttt{quantile},  & \texttt{cummax}, \\
localisation & & \texttt{which.max} & \texttt{range} & \texttt{cummin} \\ 
 & & \texttt{which.min} & \texttt{summary} & \texttt{rank} \\ \hline
centralité & & \texttt{mean}, \texttt{median} & \texttt{summary} & \\ \hline
variabilité & & \texttt{sd} & \texttt{var}, \texttt{cov}, \texttt{cor} & \\ \hline
fréquences & & &  \texttt{table}, \texttt{xtabs}, \texttt{ftable} & \\ \hline

\end{tabular}
\end{scriptsize}
\end{columns}

\end{frame}

\end{document}
